\nsection{Tents \& Trees}

Tents is a logic puzzle in which you work out where to place tents next to trees
in a field:

The rules are explained at:

\wwwurl{https://www.puzzle-tents.com}

\begin{itemize}
\item Pair each tree with a tent adjacent horizontally or vertically (4-connected).
This should be a 1 to 1 relation.
\item Tents never touch each other even diagonally (8-connected)
\item The clues outside the grid indicate the number of tents on that row/column.
\end{itemize}
\noindent The game consists of a {\bf square} board representing a field where every
tile may be a `tree', a `tent', `grass' or unknown.
Trees are represented by the character~`*', tents using~`+', grass with~`.' and unknowns as~`~'.
An example of a completed puzzle is shown here:
\begin{center}
\begin{tikzpicture}
\matrix[matrix of nodes,nodes={draw=black, anchor=center, minimum size=.6cm,fill=ocre!30}, column sep=-\pgflinewidth, row sep=-\pgflinewidth, , execute at empty cell={\node[draw=black,text=black,fill=ocre!5]{};} ] (A) {
\node[fill=white,draw=white]{~};&
\node[fill=white,draw=white]{1};&
\node[fill=white,draw=white]{2};&
\node[fill=white,draw=white]{0};&
\node[fill=white,draw=white]{1};&
\node[fill=white,draw=white]{2};&
\node[fill=white,draw=white]{1};\\
\node[fill=white,draw=white]{2};&*& & & & &*\\
\node[fill=white,draw=white]{0};& & & & &*& \\
\node[fill=white,draw=white]{1};& & & & & & \\
\node[fill=white,draw=white]{1};& & &*& & &*\\
\node[fill=white,draw=white]{1};& & & & & & \\
\node[fill=white,draw=white]{2};& &*&*& & & \\
};
\end{tikzpicture}
\hspace*{1cm}
\begin{tikzpicture}
\matrix[matrix of nodes,nodes={draw=black, anchor=center, minimum size=.6cm,fill=ocre!20}, column sep=-\pgflinewidth, row sep=-\pgflinewidth, , execute at empty cell={\node[draw=black,text=black,fill=ocre!20]{.};} ] (A) {
\node[fill=white,draw=white]{~};&
\node[fill=white,draw=white]{1};&
\node[fill=white,draw=white]{2};&
\node[fill=white,draw=white]{0};&
\node[fill=white,draw=white]{1};&
\node[fill=white,draw=white]{2};&
\node[fill=white,draw=white]{1};\\
\node[fill=white,draw=white]{2};&*&+&.&.&+&*\\
\node[fill=white,draw=white]{0};&.&.&.&.&*&.\\
\node[fill=white,draw=white]{1};&.&.&.&.&+&.\\
\node[fill=white,draw=white]{1};&.&+&*& &.&*\\
\node[fill=white,draw=white]{1};&.&.&.&.&.&+\\
\node[fill=white,draw=white]{2};&+&*&*&+&.&.\\
};
\end{tikzpicture}
\end{center}
\noindent (Left) Puzzle as defined - only the trees are given to begin
with, all other tiles are unknown. There are $7$ tents to be found,
the number of them in total per row and column is also shown. (Right)
The solved puzzle after applying some simple logic.


\begin{exercise}

We will implement five rules which will allow {\bf some}, but {\bf not}
all, puzzles to be solved:

\begin{enumerate}

\item If a row or column already has the correct number of tents, all
unknowns $\rightarrow$ grass.

\item Any unknown cell not 4-connected to a tree $\rightarrow$ grass.

\item If the number of unknown cells equals the number of tents required,
all unknowns $\rightarrow$ tents.

\item Any unknown cell 8-connected to a tent $\rightarrow$ grass.

\item For any tree and its 4-connected cells, if there is exactly one
unknown and zero tents, then that unknown $\rightarrow$ tent.

\end{enumerate}

\noindent Some skeleton code may be found in~:
\wwwurl{https://github.com/csnwc/Exercises-In-C} \noindent then navigate
into \verb^Code/Week5/TentsTrees^.

You may assume that:
\begin{itemize}

\item The size of the board is always square. Its exact size is {\tt
\#defined} (via the Makefile).

\item You will never be expected to solve a board any larger than $9x9$.
\end{itemize}

Complete the file {\bf tt.c} which, along with
my files {\em driver.c} and {\em tentstrees.h}, implements some important
functionality necessary for a game of this type.

\noindent My file {\em tentstrees.h} contains the function definitions
which you'll have to implement in your {\bf tt.c} file. My file
{\em driver.c} contains the \verb^main()^ function to act as a driver to
run the code. Your file will contain many other functions as well as those
specified, so you'll wish to test them as normal using a \verb^test()^
function.

\noindent If all of these files are in the same directory, you can
compile them using the \verb^Makefile^ given.

\noindent The functions you need to complete include:

\verb^inputboard()^ - this takes a string corresponding to the initial
state of the board (including row seperator characters `|',
two strings containing the number of tents in each
column/row, and a pointer to the board state to be filled in. It returns
a Boolean, true if all is OK, false if the number of characters in the
input is incorrect, or if the total number of tents in the rows is
different to that in the columns, or if incorrect characters have been
used - see the assertions for more examples of failure.

\verb^solve()^ - this takes a pointer to the board state, and returns
a Boolean, depending on whether the five rules, applied repeatedly,
can solve this board or not.

\verb^board2str()^ - this takes a copy of the board state, and
copies this board into the string, in the same format as required for
\verb^inputboard^. It returns a Boolean, with a NULL string causing
a failure.

\noindent Hints:
\begin{itemize}

\item To begin with use your own, simpler driver file - mine makes sense
once everything is working, but may seem complex to begin with.

\item Your \verb^tt.c^ file should contain many other sub-functions
which are used by the major ones specified. You can put anything in this
file, provided it still compiles as specified.

\item Do not alter or resubmit {\em driver.c}, {\em Makefile} or
{\em tentstrees.h} - my original versions (and variants) will be
used to compile the {\em tt.c} file that you write.

\item If you have questions such as {\em "What happens if a NULL pointer is passed
to solve()"} - I'm happy for you to make your own judgement, provided that your code
does something graceful i.e. cannot be crashed.

\end{itemize}

\end{exercise}
